\section{Test Document}



    \bgroup
    \def\arraystretch{1.5}
    \setlength\arrayrulewidth{1.2pt}
    \color{textgray}
    \begin{xltabular}{\textwidth}{|X|X|M{1.8cm}|M{1.3cm}|M{1.3cm}|}

\caption{Implementation Phase – Requirements - Functional} \label{tab:Requirements - Functional} \\

\arrayrulecolor{linegray}\hline \rowcolor{lightgray} \multicolumn{1}{|c|}{\color{default}\textbf{Requirement}} & \multicolumn{1}{c|}{\color{default}\textbf{User Story}} & \multicolumn{1}{c|}{\color{default}\textbf{Importance}} & \multicolumn{1}{c|}{\color{default}\textbf{Jira Issue}} & \multicolumn{1}{c|}{\color{default}\textbf{Status}}\\ \hline

 \endfirsthead 
 \multicolumn{5}{c}%
{\tablename\ \thetable{} -- continued from previous page} \\ \hline \multicolumn{1}{|c|}{\textbf{Requirement}} & \multicolumn{1}{c|}{\textbf{User Story}} & \multicolumn{1}{c|}{\textbf{Importance}} & \multicolumn{1}{c|}{\textbf{Jira Issue}} & \multicolumn{1}{c|}{\textbf{Status}}\\ \hline 
\endhead \hline 
\multicolumn{5}{|r|}{{Continued on next page}} \\ \hline 
\endfoot

\hline 
 \endlastfoot 
\textbf{1.} A maximum of 5 bird Avatar-elements shall be displayed in the main page. The most recent bird shall be at the front of the list. \newline \textbf{2.} An Indicator shall signal if more than 5 birds were added. & As a user I want to see an overview of my added birds in the main page & \tcbox[colframe=TAGred, colback=TAGred]{\textbf{\footnotesize HIGH}} & {\color{purpleT}\ttfamily DLMCSPSE-224} & \tcbox[colframe=TAGgreen, colback=TAGgreen]{\textbf{\footnotesize DONE}} \\ \hline 
  \textbf{1.} Birds shall be addable by providing the initial XYZ-coordinates. If no coordinate or a NaN-Value is provided, it shall be interpreted as 0. \newline \textbf{2.} A bird shall not be addable if its distance is greater than 100m from the plane. If such values are provided, the user shall be informed via a Toast-Message. \newline \textbf{3.} A bird shall be removable via a simple click on an X-icon, coloured in the birds highlight colour, on/adjacent to the bird’s avatar. & As a user I want to easily add and remove nearby threats without leaving the main page & \tcbox[colframe=TAGred, colback=TAGred]{\textbf{\footnotesize HIGH}} & {\color{purpleT}\ttfamily DLMCSPSE-225} & \tcbox[colframe=TAGgreen, colback=TAGgreen]{\textbf{\footnotesize DONE}} \\ \hline 
  \textbf{1.} A click on the birds avatar shall select the corresponding bird. \newline \textbf{2.} The fact that a bird is selected shall be visible in the UI. \newline \textbf{3.} Only one bird shall be selectable at a time. \newline \textbf{4.} A second click on a selected bird shall deselect said bird. \newline & As a user I want to be able to highlight a specific bird to make this specific bird stand out in the UI. & \tcbox[colframe=TAGred, colback=TAGred]{\textbf{\footnotesize HIGH}} & {\color{purpleT}\ttfamily DLMCSPSE-226} & \tcbox[colframe=TAGgreen, colback=TAGgreen]{\textbf{\footnotesize DONE}} \\ \hline 
  \textbf{1.} The user shall be able to translate the plane in space by adding values to the current position of the planes XYZ-axis. \newline \newline \textbf{2.} If no coordinate or a NaN-Value is provided, it shall be interpreted as 0. \newline \newline \textbf{3.} The plane shall not be able to be translated below ground. \newline \textbf{a.}  If this would be the case, the plane shall not be moved and an error shall be displayed. & As a user I want to move the plane in space. & \tcbox[colframe=TAGred, colback=TAGred]{\textbf{\footnotesize HIGH}} & {\color{purpleT}\ttfamily DLMCSPSE-228} & \tcbox[colframe=TAGgreen, colback=TAGgreen]{\textbf{\footnotesize DONE}} \\ \hline 
  \textbf{1.} The plane shall not be distorted → the viewport shall remain a cube. \newline\newline \textbf{2.} The view shall always be centred around the plane. \newline\newline \textbf{3.} The side length of aforementioned cube shall be the distance to the furthest threat and at least 10. & As a user I always want to see the plane centred and as close as possible. & \tcbox[colframe=TAGred, colback=TAGred]{\textbf{\footnotesize HIGH}} & {\color{purpleT}\ttfamily DLMCSPSE-229} & \tcbox[colframe=TAGgreen, colback=TAGgreen]{\textbf{\footnotesize DONE}} \\ \hline 
  All birds shall be displayed as a point in space in the colour of the corresponding bird. & As a user I want to see my added birds in the 3D-Diagram in the main page & \tcbox[colframe=TAGred, colback=TAGred]{\textbf{\footnotesize HIGH}} & {\color{purpleT}\ttfamily DLMCSPSE-231} & \tcbox[colframe=TAGgreen, colback=TAGgreen]{\textbf{\footnotesize DONE}} \\ \hline 
  \textbf{1.} The user shall be able to translate a bird in space.\newline\newline \textbf{2.} A Bird cannot be below ground. & As a user I want to be able to move a threat in space & \tcbox[colframe=TAGred, colback=TAGred]{\textbf{\footnotesize HIGH}} & {\color{purpleT}\ttfamily DLMCSPSE-232} & \tcbox[colframe=TAGgreen, colback=TAGgreen]{\textbf{\footnotesize DONE}} \\ \hline 
  The user shall be able to rotate the plane by providing Euler Angles in ZYX order.\newline \newline It shall be possible to enter angles greater than 360° \newline \newline If no angle or a NaN-Value is provided, it shall be interpreted as 0. & As a user I want to be able to rotate the plane. & \tcbox[colframe=TAGred, colback=TAGred]{\textbf{\footnotesize HIGH}} & {\color{purpleT}\ttfamily DLMCSPSE-233} & \tcbox[colframe=TAGgreen, colback=TAGgreen]{\textbf{\footnotesize DONE}} \\ \hline 
  \textbf{1.} The user shall be able to navigate a page displaying every bird as a card component detailing its relation to the plane. Listed shall be the following information \newline \textbf{a} name \newline \textbf{b} distance to plane \newline \textbf{c} Horizontal Distance to plane \newline \textbf{d} angle from plane 2 bird (azimuth and elevation)\newline \textbf{e} altitude \newline \textbf{f} an indication of the direction and horizontal distance as an icon from the viewpoint of the user. \newline \newline \textbf{2.} The information on the main page shall not be lost when navigation to and back from this page. & As a user I want to see a detailed view of all the birds and their relation to the plane. & \tcbox[colframe=TAGred, colback=TAGred]{\textbf{\footnotesize HIGH}} & {\color{purpleT}\ttfamily DLMCSPSE-240} & \tcbox[colframe=TAGgreen, colback=TAGgreen]{\textbf{\footnotesize DONE}} \\ \hline 
  The planes X-Axis shall represent the head of the plane and the ground X-Axis shall be treated as North (positive) and South (negative) in the compass component. & As a user I want to see a 2D “Compass-like” Representation of the planes rotation. & \tcbox[colframe=TAGred, colback=TAGred]{\textbf{\footnotesize HIGH}} & {\color{purpleT}\ttfamily DLMCSPSE-244} & \tcbox[colframe=TAGgreen, colback=TAGgreen]{\textbf{\footnotesize DONE}} \\ \hline 
  The compass component shall indicate the birds' current, relative to the plane, position.\newline \textbf{Horizontal Distance \textless = 14:} \newline A Bird Icon shall indicate the position. \newline \newline \textbf{Horizontal Distance \textgreater 14:} \newline A Triangle shall indicate the direction of the bird & As a user I want to see the birds displayed in their relative distance to the plane in the compass component. & \tcbox[colframe=TAGred, colback=TAGred]{\textbf{\footnotesize HIGH}} & {\color{purpleT}\ttfamily DLMCSPSE-245} & \tcbox[colframe=TAGgreen, colback=TAGgreen]{\textbf{\footnotesize DONE}} \\ \hline 
  The home page shall contain an article explaining the idea and concept behind this app & As a user I want to read an explanation on what I am seeing here & \tcbox[colframe=TAGred, colback=TAGred]{\textbf{\footnotesize HIGH}} & {\color{purpleT}\ttfamily DLMCSPSE-249} & \tcbox[colframe=TAGgreen, colback=TAGgreen]{\textbf{\footnotesize DONE}} \\ \hline 
  The birds name shall be displayed in a Tooltip with the birds color as the background color over the birds avatar. & As a user I want to see the birds name when I hover over it, to later identify it easier in the Card View & \tcbox[colframe=TAGgray, colback=TAGgray]{\textbf{\footnotesize LOW}} & {\color{purpleT}\ttfamily DLMCSPSE-227} & \tcbox[colframe=TAGgreen, colback=TAGgreen]{\textbf{\footnotesize DONE}} \\ \hline 
  The footer shall include a link to a page displaying the legal information. \newline All transformations that have been made, up to the point of navigating away, shall not be lost when navigating back to the main page. & An Impressum is mandated by law to be included on every published medium in German speaking countries. & \tcbox[colframe=TAGgray, colback=TAGgray]{\textbf{\footnotesize LOW}} & {\color{purpleT}\ttfamily DLMCSPSE-236} & \tcbox[colframe=TAGgreen, colback=TAGgreen]{\textbf{\footnotesize DONE}} \\ \hline 
  - & As a user I would like to see the selected birds name in the “Move Bird”-Component. So it is absolutely clear which bird I’m translating. & \tcbox[colframe=TAGgray, colback=TAGgray]{\textbf{\footnotesize LOW}} & {\color{purpleT}\ttfamily DLMCSPSE-242} & \tcbox[colframe=TAGgreen, colback=TAGgreen]{\textbf{\footnotesize DONE}} \\ \hline 
  - & As a user I would like to set the center of my angle and distance calculation to the center of the plane & \tcbox[colframe=TAGgray, colback=TAGgray]{\textbf{\footnotesize LOW}} & {\color{purpleT}\ttfamily DLMCSPSE-247} & \tcbox[colframe=TAGgreen, colback=TAGgreen]{\textbf{\footnotesize DONE}} \\ \hline 
  Show two Arrows at ground Level indicating N and E & As a user I would like to easier see North and East in the Diagram & \tcbox[colframe=TAGgray, colback=TAGgray]{\textbf{\footnotesize LOW}} & {\color{purpleT}\ttfamily DLMCSPSE-248} & \tcbox[colframe=TAGgreen, colback=TAGgreen]{\textbf{\footnotesize DONE}} \\ \hline 
  \textbf{1.} The name shall be displayed in the birds colour. \newline\newline \textbf{1.} The layout shall be \enquote{\textit{\textless Name\textgreater \textbackslash n$(X,Y,Z)$}} & As a user I would like to see the name of the bird as I’m hovering over it in the diagram. & \tcbox[colframe=TAGyellow, colback=TAGyellow]{\textbf{\footnotesize MEDIUM}} & {\color{purpleT}\ttfamily DLMCSPSE-230} & \tcbox[colframe=TAGgreen, colback=TAGgreen]{\textbf{\footnotesize DONE}} \\ \hline 
  A page displaying the math behind these transformations shall be accessibe via the main nav bar. & As a user I want to understand the math behind these transformations & \tcbox[colframe=TAGyellow, colback=TAGyellow]{\textbf{\footnotesize MEDIUM}} & {\color{purpleT}\ttfamily DLMCSPSE-234} & \tcbox[colframe=TAGgreen, colback=TAGgreen]{\textbf{\footnotesize DONE}} \\ \hline 
  The initial values of the plane shall be restoreable with the click of a button. & As a user I want to reset the plane to its default position without having to reload the page. & \tcbox[colframe=TAGyellow, colback=TAGyellow]{\textbf{\footnotesize MEDIUM}} & {\color{purpleT}\ttfamily DLMCSPSE-235} & \tcbox[colframe=TAGgreen, colback=TAGgreen]{\textbf{\footnotesize DONE}} \\ \hline 
  The rendering of the coordinate system shall be conditional. \newline But the marble shall stay and not be rendered conditionally. \newline The user shall toggle this condition via an switch component in the main page. & As a user I want to be able to hide and show the “Pilot” and plane coordinate system at will. & \tcbox[colframe=TAGyellow, colback=TAGyellow]{\textbf{\footnotesize MEDIUM}} & {\color{purpleT}\ttfamily DLMCSPSE-239} & \tcbox[colframe=TAGgreen, colback=TAGgreen]{\textbf{\footnotesize DONE}} \\ \hline 
  Display a shadow of the plane on the ground. \newline The sun shall be assumed to be perpendicular to the ground. & As a user I would like to a shadow of the plane to get a better sense of perspective. & \tcbox[colframe=TAGyellow, colback=TAGyellow]{\textbf{\footnotesize MEDIUM}} & {\color{purpleT}\ttfamily DLMCSPSE-241} & \tcbox[colframe=TAGgreen, colback=TAGgreen]{\textbf{\footnotesize DONE}} \\ \hline 
  These values shall only be rendered, iff a bird is selected. & As a user it would be nice to see the angle and distance to the selected bird in the compass component. & \tcbox[colframe=TAGyellow, colback=TAGyellow]{\textbf{\footnotesize MEDIUM}} & {\color{purpleT}\ttfamily DLMCSPSE-246} & \tcbox[colframe=TAGgreen, colback=TAGgreen]{\textbf{\footnotesize DONE}} \\ \hline 
  The NavBar shall contain a link to a page showing a bigger view of the plane scene. & As a user I would like to be able to see a bigger image of the plane scene. & \tcbox[colframe=TAGyellow, colback=TAGyellow]{\textbf{\footnotesize MEDIUM}} & {\color{purpleT}\ttfamily DLMCSPSE-250} & \tcbox[colframe=TAGgreen, colback=TAGgreen]{\textbf{\footnotesize DONE}} \\ \hline 
  - & As a user I would like to see visual pleasing 404 Page if i entered the wrong url. & \tcbox[colframe=TAGgray, colback=TAGgray]{\textbf{\footnotesize LOW}} & {\color{purpleT}\ttfamily DLMCSPSE-243} & \tcbox[colframe=TAGred, colback=TAGred]{\textbf{\footnotesize M2FR}} \\ \hline 
  The math explanation page shall contain at least one diagram explaining the azimuth and elevation angles along with imporant sides aiding in calculation & As a user I want to see an explanation of the mentioned azimuth and elevation angles & \tcbox[colframe=TAGyellow, colback=TAGyellow]{\textbf{\footnotesize MEDIUM}} & {\color{purpleT}\ttfamily DLMCSPSE-237} & \tcbox[colframe=TAGred, colback=TAGred]{\textbf{\footnotesize M2FR}} \\ \hline 
  The math explanation page shall contain at least one diagram visualising the planes axes. & As a user I want to see an explanation of the planes coordinate system & \tcbox[colframe=TAGyellow, colback=TAGyellow]{\textbf{\footnotesize MEDIUM}} & {\color{purpleT}\ttfamily DLMCSPSE-238} & \tcbox[colframe=TAGred, colback=TAGred]{\textbf{\footnotesize M2FR}} \\ \hline 
  
\end{xltabular} 
 \egroup 
 \color{default}