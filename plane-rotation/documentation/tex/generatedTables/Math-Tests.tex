
    \bgroup
    \def\arraystretch{1.5}
    \setlength\arrayrulewidth{1.2pt}
    \color{textgray}
    \begin{xltabular}{\textwidth}{|M{1.2cm}|X|X|M{3cm}|}

\caption*{} \label{tab:Requirements - Functional} \\

\arrayrulecolor{linegray}\hline \rowcolor{lightgray} \multicolumn{1}{|c|}{\color{default}\textbf{ID}} & \multicolumn{1}{c|}{\color{default}\textbf{Function/Method}} & \multicolumn{1}{c|}{\color{default}\textbf{Description}} & \multicolumn{1}{c|}{\color{default}\textbf{Coverage}}\\ \hline

 \endfirsthead 
 \multicolumn{4}{c}%
{\tablename\ \thetable{} -- continued from previous page} \\ \hline \multicolumn{1}{|c|}{\textbf{ID}} & \multicolumn{1}{c|}{\textbf{Function/Method}} & \multicolumn{1}{c|}{\textbf{Description}} & \multicolumn{1}{c|}{\textbf{Coverage}}\\ \hline 
\endhead \hline 
\multicolumn{4}{|r|}{{Continued on next page}} \\ \hline 
\endfoot

\hline 
 \endlastfoot 
\textbf{ F1 } & {\ttfamily azimuthAngle(A: Point, center?: Point): number} & Calculates the angle measure (in radians, with $-\pi \textless \theta \le \pi$ between the positive x-axis and the ray from the origin to the point {\ttfamily(x,y)} in the Cartesian plane.\newline If a center point is provided, the angle is calculated from the new center position. & \tcbox[colframe=TAGgreen, colback=TAGgreen]{\textbf{\footnotesize 100\%}} \\ \hline 
  
\end{xltabular} 
 \egroup 
 \color{default}